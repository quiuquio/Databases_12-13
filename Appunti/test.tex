\documentclass[10pt]{report}
\usepackage[italian]{babel}
\usepackage[T1]{fontenc}
\usepackage[utf8]{inputenc}


\begin{document}
\title{Algebra relazionale e gestione dei valori NULL}
\author{Lezioni tenute dalla Prof. Raffalella Gentilini
	\and Appunti di Robert Parcus
	\and \texttt{betoparcus@gmail.com}}
\date{10 ottobre 2012}
\maketitle
2012/10/16 - Algebra relazionale 

\section{La divisione}

Indicato pi\'u intuitivamente controparte della quantificazione universale. \\
Esempio: Date due relazioni \\*

ISCRIZIONE:
\begin{tabular}{|c|c|}
\hline 
matricola & id\_corso \\ 
\hline 
123 & BD \\ 
\hline 
283 & BD \\ 
\hline 
123 & INF \\ 
\hline 
123 & MAT \\ 
\hline 
283 & MAT \\ 
\hline 
375 & BD \\ 
\hline 
283 & INF \\ 
\hline 
\end{tabular} 
CORSO:
\begin{tabular}{|c|}
\hline 
id\_corso \\ 
\hline 
BD \\ 
\hline 
INF \\ 
\hline 
MAT \\ 
\hline 
\end{tabular} 

$R \leftarrow ISCRIZIONE \div CORSO $  \\*
(mi dar\'a   le matricole iscritte a tutti i corsi appartenenti a CORSO)

R: \begin{tabular}{|c|c|}
\hline 
123 & 283 \\
\hline 
\end{tabular} 
\\*
Vediamo come si pu\'o derivare la divisione dalle altre operazioni che conosciamo:\\*

Definizione $[DIVISIONE]$ : Siano $R1(x), R2(y)$ due schemi di relazione t.c. $Y$ appartiene a $X$ e siano
$r1, r2$ due istanze di $R1R2$. L'operatore divisione produce una relazione le cui tuple, se estese con
una qualunque tupla della seconda relazione producono una tupla di $R1$.\\
$r1 \div r2 = \lbrace\, t\, \vert\, \forall \, t' \in r2, \; t \cup t \in r1 \,\rbrace$ \\*


La divisione $r1 \div r2$ \'e  dunque un operatore definito sullo schema con attributi $X - Y$.
E' un operatore derivato e pu\'o essere definito come:\\*
$T1 \leftarrow \Pi _{x-y}(R1)$\\*
$T2 \leftarrow \Pi _{x-y}(R2 \times T1)-R1)$\\*

Sull'esempio di prima:\\*
$T1 \leftarrow \Pi _{\textit{matricola}}(\textit{Iscrizione})$\\*
$T2 \leftarrow \Pi _{\textit{matricola}}((T1 \times \textit{corso})-R1)$\\*
$R \leftarrow \textit{T1T2}$\\*


\section{ALGEBRA RELAZIONE E VALORI NULL}

La presenza di valori nulli nelle istanze di una base di dati richiede un'estensione della semantica degli operatori dell'algebra relazionale. L'approccio tradizionale  (usato anche nei DBMS commerciali ed in SQL) \'e quello di considerare una logica a 3 valori per la valutazione delle formule proposizionali e quei nodi degli operatori di Selezione e Join.\\*

\textbf{Proiezione, unione, differenza con valori nulli:\\}
\begin{center}
Continuano a comportarsi usualmente. L'uguaglianza tra NULL \'e un livello sintattico e due tuple sono sono uguali anche se ci sono valori NULL.\\*
\end{center}

Esempio:

IMPIEGATI:
	\begin{tabular}{|c|c|c|}
	\hline 
	codice & nome & ufficio \\ 
	\hline 
	123 & Rossi & A12 \\ 
	\hline 
	231 & Verdi & NULL \\ 
	\hline 
	373 & Verdi & A27 \\ 
	\hline 
	435 & Verdi & NULL \\ 
	\hline 
	\end{tabular} 
Responsabili:
	\begin{tabular}{|c|c|c|}
	\hline 
	codice & nome & ufficio \\ 
	\hline 
	123 & Rossi & A12 \\ 
	\hline 
	NULL & NULL & NULL \\ 
	\hline 
	435 & Verdi & A27 \\ 
	\hline 
	\end{tabular} 

manca roba...

\textbf{Selezione e valori nulli:\\}

Per la selezione, il problema \'e stabilire se in presenza di valori NULL un predicato \'e vero o meno.\\*

IMPIEGATI:
\begin{tabular}{|c|c|c|c|}
\hline 
codice & nome & ufficio & risultato \\ 
\hline 
123 & Rossi & A12 & OK \\ 
\hline 
231 & Verdi & NULL & ??? \\ 
\hline 
373 & Verdi & A27 & OK \\*
\hline 
\end{tabular}
\\
$\Pi _{ \textit{ufficio='A12'} }(\textit{IMPIEGATI})$\\*

La seconda tupla fa parte del risultato?\\
\textit{Non si pu\'o sapere.}\\*
Vanno definite politiche di gestione del NULL:\\*

Per verificare predicati come quelli dell'esempio precedente, si introduce una logica a 3 valori, dove oltre al valore V ed F si ha un terzo valore '?'.\\*

\begin{tabular}{|c|c|}
\hline 
a=a & V \\ 
\hline 
a=b & F \\ 
\hline 
a=NULL & ? \\ 
\hline
NULL=NULL & ? \\
\hline 
a!=a & F \\ 
\hline 
a!=b & V \\ 
\hline 
a!=null & ? \\ 
\hline 
null!=null & ? \\ 
\hline 
\end{tabular} 

Queste tabelle di verit\'a possono essere estese ad altri tipi di confronto,. Se D \'e un dominio su cui \'e definito un ordinamento < (>,>=, <=), allora per 'e' appartenente {<,> =} il confronto x e y \'e indefinito se x o y sono NULL.
Le tabelle di verit\'a   nella logica a 3 valori per gli operatori booleani sono 
NOT:
\begin{tabular}{|c|c|}
\hline 
V & F \\ 
\hline 
F & V \\ 
\hline 
? & ? \\ 
\hline 
\end{tabular} 
AND:
\begin{tabular}{|c|c|}
\hline 
V & V F ? \\ 
\hline 
F & F F F \\ 
\hline 
? & ? F ? \\ 
\hline 
\end{tabular} 
OR:
\begin{tabular}{|c|c|}
\hline 
V & V V V \\ 
\hline 
F & V F ? \\ 
\hline 
? & V ? ? \\*
\hline 
\end{tabular} 

Una selezione $G(f)$ produce le sole tuple per cui la condizione di selezione risulti true.\\
Esempio:
	$\sigma _{ \textit{ufficio='A12' OR uffizio != 'A12'}} (\textit{IMPIEGATI})$
Non restituisce tutte le tuple!!!
Per lavorare esplicitamente con i valori nulli si introducono le condizioni  atomiche
	- a IS NULL
	- a IS NOT NULL
Selezione in ufficio= 'A12' || ufficio != 'A12' || ufficio IS NULL (IMPIEGATI)

JOIN con valori NULL:

Valgono le stesse regole di valutazione per F utilizzate nella selezione. Il JOIN NATURALE non combina due tuple se queste hanno entrambe valori nulli su un attributo in comune.

Esempio:
	Impiegati:
		codice 123 	231 	272 	435
		nome 	Rossi	Verdi	''	''
		ufficio	A12 	NULL 	A27 	NULL
	Responsabili:
		ufficio A12 	A27 	NULL
		codice	123	NULL	231
		
		
\section{esercizi 18/10}
Consideriamo la seguente porzione di BD:\\*
\newline
FORNITORE(\underline{CodForn}, nome, citt\'a)\\*
PRODOTTI(\underline{CodProd}, nome, marca, modello)\\*
CATALOGO(\underline{CodForn},\underline{CodProd}, Costo)\\*

2) Trovare i nomi dei fornitori che distribuiscono prodotti di marca IBM:

\begin{center}
$ T _{1} \; \leftarrow \; \sigma _{marca=IBM} \, ( PRODOTTI \bowtie CATALOGO) $
\end{center}

3)Trovare i codici di tutti i prodotti che sono forniti da almeno 2 fornitori:\\*
\underline{Idea:} Concateniamo due copie di catalogo su valori uguali dell'attributo codice prodotto e troviamo le tuple su cui i fornitori sono diversi.

\begin{center}
$COPIACATALOGO(codforn1, codforn2) \leftarrow \Pi _{codforn, prod} (CATALOGO) $\\
$ T _{1} \; \leftarrow \; COPIACATALOGO \bowtie CATALOGO $\\
$RIS \leftarrow \Pi ( \sigma _{codforn1, codforn2} (T1)$\\
\end{center}

4)Nomi dei fornitori che distribuiscono tutti i prodotti:\\
\begin{center}
$T1 \leftarrow ( \Pi _{codFor,codProd}(CATALOGO)) \div (\Pi _{codProd}(PRODOTTI))$\\
$RIS \leftarrow \Pi ( T_{1} \bowtie FORNITORE )$\\
\end{center}
Oppure, senza usare la divisione:\\
	//Tutte le coppie fornitore e prodotto
\begin{center}
$T_{1} \leftarrow	 \Pi _{codforn}(CATALOGO) \times \P _{codProd} (PRODOTTO)$\\
$	//CodForn \notin T2 se cod_Forn fornisce tutti i prodotti.$\\
$T_{2} \leftarrow \Pi(T_{1} - \Pi _{codForn,codProd} (CATALOGO))$\\
$RIS \leftarrow \Pi _{codForn}(CATALOGO) - T2 $\\
\end{center}


STADIO(\underline{nome}, citt\'a, capienza)
PARTITA(\underline{stadio, dato, ora}, squadra1, squadra2)
SQUADRA(\underline{nazione}, continente, livello)

1) Determinare gli stadi in cui non gioca alcuna squadra africana:\\*

\begin{center}
$AFRICA \leftarrow \Pi _{(nazione)}(\sigma _{(continente \neq africa)}(SQUADRA)))$
$BAD \leftarrow \Pi _{(stadio)}(PARTITA \bowtie _{(nazione = squadra1 AND nazione = squadra2)} AFRICA)$
$RIS \leftarrow \Pi _{(nome)}(STADIO) - BAD $
\end{center}

2)Determinare le squadre che incontrano soltanto squadre dello stesso livello:\\*

\begin{center}
\small
SQUADRA1\_LIVELLO(S1, S2, L1) $\leftarrow \Pi _{(squadra1, squadra2, livello)}(PARTITA \bowtie_{(squadra1 = nazione)} SQUADRA $
SQUADRA2\_LIVELLO(S1, S2, L1, l2) $\leftarrow \Pi _{(S1, S2,L1, livello)}(SQUADRA1\_LIVELLO \bowtie_{(S2 = nazione)} SQUADRA $
$BAD1 \leftarrow \Pi _{(S1)}( \sigma _{L1 \not= L2}(SQUADRE\_LIVELLO)))$\\
$BAD2 \leftarrow \Pi _{(S2)}( \sigma _{L1 \not= L2}(SQUADRE\_LIVELLO))$\\
$RIS \leftarrow \Pi _{(nazione)}(squadra) - (BAD1 \cup BAD2)  $\\
\end{center}
\normalsize

Sia dato il seguente schema relazione che descrive gli esami (obbligatori e non) in un anno ( I, II, III) 
di un indirizzo di laurea triennale: \\*
\newline
CORSO(\underline{cod\_c}, nome, CFU, SSD)\\*
INDIRIZZO(\underline{COD\_1}, titolo)\\*
COMPOSIZIONE(\underline{COD\_I, COD\_C}, tipo, anno)\\*
\newline

1)Determinare i titoli dei corsi che in almeno un indirizzo possono essere collocati indifferentemente in ciascuno dei tra anni:\\

\begin{center}
$ANNO1(codc1,codI1) \leftarrow \Pi _{(codc,codI)} \sigma _{(anno = 1)}(COMPOSIZIONE)$\\
$ANNO2(codc2,codI2) \leftarrow \Pi _{(codc,codI)} \sigma _{(anno = 2)}(COMPOSIZIONE)$\\
$ANNO3(codc3,codI3) \leftarrow \Pi _{(codc,codI)} \sigma _{(anno = 3)}(COMPOSIZIONE)$\\
$GOOD \leftarrow ( ANNO1 \bowtie _{(codc1 = codc2 AND codI1 = codI2)} ANNO2) \bowtie _{(codc2 = codc3 AND codI2 = codI3)} ANNO3)$\\
$RISULTATO \leftarrow \Pi _{(TITOLO)}(GODD \bowtie _{(codc1 = codc)}CORSO)$\\
\end{center}

2)Determinare i titoli degli indirizzi che prevedono lo stesso insieme di esami obbligatori dell'indirizzo 'Sistemi'://

//Corsi obligatori dell'indirizzo sistemi
\begin{center}
$OBBSIST \leftarrow \Pi _{(codc)} (\sigma _{(TITOLO = sistemi AND tipo = obb)}(INDIRIZZO \bowtie COMPOSIZIONE))$
\end{center}
//Tutti i corsi obbligatori
$OBB \leftarrow \Pi _{(codI, codc)} (\sigma _{(tipo = obb)} (COMPOSIZIONE))$
//indirizzi che non voglio nel risultato
$\Pi in cod i ((pi in codi(indirizzi) x OBBSIST) - OBB) \cup \Pi _{codI} (OBB - (\Pi _{codI}(INDIRIZZI) \times OBBSIST))$






\end{document}